\section{Conclusion and Research Directions}
\label{sec:conclusion}

In this paper, we presented FEVER, an approach to automatically extract and build a feature-centered representation of changes
in commits affecting the implementation of features in highly variable software systems. 
FEVER retrieves commits from versioning systems and, using model-based differencing, 
extracts detailed information on the changes, to finally combine them into feature-oriented changes. 
We applied this approach to the Linux kernel and used the constructed dataset to evaluate its accuracy 
in terms of complex change representation.
We showed that we were able to accurately extract and integrate changes from various artefacts in 87.2\%  of the studied commits.

Our exploratory study of co-evolution in the Linux kernel showed that  co-evolution of artefacts during feature evolution does occur, 
but, over a single release, most features
only evolve through their implementation. A majority of developers focus only on the feature implementation
and, over the course of a release, only few modify variability spaces beyond the implementation.
We also found that, while co-evolution of artefacts occurs in every release, they account for less than 22\% of feature evolution scenarios,
and only 11\% of authors will modify all variability spaces over the course of a release, but over time, 69,51\% of authors will only modify the implementation of features without affecting the variability model or feature-asset mapping.

Through this work we make the following key contributions:
\begin{itemize}
\item a model-based approach to extract and consolidate feature changes across variability spaces
\item an model of feature-oriented changes, focusing on the co-evolution of artefacts in different variability spaces during feature evolution
\item an evaluation of the FEVER prototype implementation, as well as a evaluation of the improvement with respect to its previous installment
\item a quantitative study describing the frequency of artefact co-evolution in the context of feature
changes from a feature perspective, and authorship perspective
\item several examples demonstrating the potential usage and value of the data gathered by such an approach for developers and researchers working
on configurable software systems
\item an implementation of FEVER as well as the full dataset, available for download\footnote{http://swerl.tudelft.nl/bin/view/NicolasDintzner/WebHome}
\end{itemize}

There several ways in which the FEVER approach and its evaluation can be further enhanced in the future. 
First, let us consider potential improvement regarding the approach itself.
At a variability model level, one could consider extracting semantic changes rather than
syntactic changes as suggested by the work of Rothberg et al. \citep{rothberg_feature_2016}.
Efficient semantic differencing on a variability model as large as the Linux kernel V.M. is a challenging task.
Moreover, given the potential size of a configuration of the Linux kernel, \ie thousands of features,
one would have to consider how to present this information in a way that can be useful to a human developer, 
making this an interesting research challenge.
Regarding mapping changes, the current FEVER approach captures only change information contained within 
changed Makefiles. A more precise approach would be to capture the exact presence condition of assets, 
rather than the main features participating in that condition.
Changes in the presence conditions will require a computationally intensive process, and the output 
might be difficult to interpret by a human. This is a direction we did not explore so far, 
but would be valuable to obtain a more sound and complete view on co-evolution changes in highly variable systems.
On a source code level,  FEVER does not consider file dependencies.
A change to an \#include statement could be a sign of changes in the relationships between features implemented within those files.
Such changes are not necessarily represented in the mapping nor the variability model.
A potential improvement of the FEVER approach would consist in taking into account file dependencies, 
and identify the nature of the symbols tying those files (functions, variable, type definitions, and so on).

To further evaluate the capabilities of the FEVER approach, we intend to apply FEVER to other systems.
Candidates for such work would be systems relying on different technologies for variability model 
description and feature-asset mappings.
The improved FEVER change meta-model and algorithm, as well as our observation on co-evolution open 
new exciting research directions.
The information captured by FEVER on changed features could prove to be useful in the domain of test case selection
for highly configurable systems.
Combining the work of Vidacs et al. \citep{vidacs_supporting_2015} on test selection in highly configurable software based on configuration and code coverage, and the work on of Soetens et al. \citep{soetens_change-based_2016} on change-based test selection supported by FEVER data 
could lead to the discovery of efficient new techniques to support testing in the context of highly variable software.

