%intro

\section{Background}
\label{sec:background}


In this section, we present how variability is supported in the Linux kernel, the different artefacts involved in its realization
and their relationships.

\subsection{Variability Model} 
A variability model (VM) formalizes the available configuration options (which we assimilate to ``features'' in this work) of a system as well as their allowed configurations \citep{kang_feature-oriented_1990}.
In the context of the Linux kernel, the VM is expressed in the Kconfig language.
An example of a feature in the Kconfig language is shown in \listingref{listing_kconfig}.
Features have at least a name (following the ``config'' keyword on line 3) and a type. 
The ``type'' attribute specifies what kind of values can be associated with a feature, which may be ``boolean'' (selected or not), 
``tristate'' (selected, selected but compiled as a module, or not selected), or a value (when the type is ``int'', ``hex'', or ``string''). 
In our example, the SQUASHFS\_FILE\_DIRECT feature is of type \textit{boolean} (line 2).
In the remainder of this work, we will refer to Boolean and tristate features simply as ``Boolean features'', while features with type ``int'', ``hex'', or ``string'', will be referred to as ``value-based features''.
The text following the type on line 3 is the ``prompt'' attribute. Its presence indicates that the feature is visible to the end user during the configuration process. 
Features can also have default values. In our example the feature is selected by default (\textit{y} on line 4).
The default value might be conditioned by an ``if'' statement. 

\begin{sloppypar}
Kconfig expresses feature dependencies using the ``depends on'' statements (see line 5). 
If the expression is satisfied, the feature becomes selectable during the configuration process. In this example, the feature SQUASHFS must be selected.
Reverse dependencies are declared using the ``select'' statement. 
If the feature is selected then the target of the ``select'' will be selected automatically as well (ZLIB\_INFLATE is the target of the ``select'' statement on line 6).
The selection occurs if the expression in the following ``if'' statement is satisfied by the current feature selection (\eg if SQUASHFS\_ZLIB is already selected).

In the context of this study, we consider additions and removals of features as well as modifications of existing ones 
\ie modifications of any attributes of a feature.

\end{sloppypar}
\vspace{-.5ex}
\begin{lstlisting}[caption=A feature declaration in Kconfig,label=listing_kconfig,language=diff]
config SQUASHFS_FILE_DIRECT
	 bool 
	 prompt "Decompress files in page cache"
	 default y
	 depends on SQUASHFS
	 selects ZLIB_INFLATE if SQUASHFS_ZLIB
	 help
	   Decompress file data in page cache.
\end{lstlisting}
\vspace{-.5ex}
To create a new kernel image, an end-user uses a configurator tool (``menuconfig'' for instance)
which reads the variability model, and presents the features to the user in a tree like structure.
At the end of the configuration process, a list of selected features is passed on to the build system 
which uses it to select artefacts and artefact fragments to include in the image before compiling them.

\subsection{Feature-Asset Mapping}
The mapping between features and assets determines which assets should be included in a product upon the selection of specific features.
In highly-configurable systems, the assets could be source code, documentation, or any other type of resources (\eg images).
In the context of this study, we consider the following types of assets : implementation artefacts (\ie source files), data artefacts (\ie hardware description files), folders, and compilation flags.
The addition of the mapping between a feature and code in a Makefile, as performed in the Linux kernel, 
is presented in \listingref{listing_makefile}.
In this example, the mapping is done between features and object files (but may link source code directly on occasion).
We use the relationship between object files and source files to identify the mapped source file.

Upon feature selection, the name of the feature used in the Makefile (symbol prefixed with CONFIG\_) will be replaced by its value. 
As a result, the compilation units (``.o'' files) will be added to different lists 
``obj-y'', ``obj-n'', and ``obj-m'' (for modules), based on the value of the macros CONFIG\_SQUASHFS\-\_FILE\_DIRECT.
Compilation units added to the list ``obj-y'' are compiled into the kernel image while those in ``obj-m'' are compiled as external modules, 
and objects in ``obj-n'' are not compiled.

Alternatively, a developer may chose to directly include ``obj-y'' list in his Makefile, in which case, 
the content of the list will be included in the compilation process as soon as the Makefile is included in the build process.
The inclusion of a Makefile in the build process may be subject to feature selection, via conditional inclusion, or more complex mechanism relying on variables and file path reconstruction.
\vspace{-.5ex}
\begin{lstlisting}[caption=Mapping between features and assets as performed in the Linux kernel,label=listing_makefile,language=diff] 
+ obj-$(CONFIG_SQUASHFS_FILE_DIRECT) += 
+	      file_direct.o page_actor.o
\end{lstlisting}
\vspace{-.5ex}
The language used to describe the mapping and implement the compilation process is a complete programming language, and the
exact mapping between feature and assets can be very complex. 
Makefiles are organized in a hierarchy, and constraints from one may affect others, leading to complex presence conditions for artefacts.

\subsection{Assets}

\begin{sloppypar}
Many types of assets exists, such as images, code, or documentation. We consider only configurable implementation assets (source files).
We focus specifically on pre-processor based variability implementation (using \#ifdef statements), which, 
despite known limitations\citep{spencer_ifdef_1992}, is still widely used today \citep{liebig_analysis_2010}.
An example of an addition of a pre-processor statement is presented in Listing \ref{listing_ifdef} 
where feature SQUASHFS\_FILE\_DIRECT is used to condition the compilation of two code blocks, 
one pre-existing (line 2 to 7) and a new one (lines 9 to 13).
As a result, based on the selection of the feature SQUASHFS\_FILE\_DIRECT during the configuration phase, only one of the two code blocks
will be included in the final product.
\end{sloppypar}
\vspace{-.5ex}
\begin{lstlisting}[caption=Creating an \#ifdef block in Linux,label=listing_ifdef,language=diff] 
+ #ifndef CONFIG_SQUASHFS_FILE_DIRECT
 static inline void *squashfs_first_page
 			(struct squashfs_page_actor *actor)
 {
 	return actor->page[0];
 }
+ #else
+ static inline void *squashfs_next_page
+			(struct squashfs_page_actor *actor)
+ {
+	return actor->squashfs_next_page(actor);
+ }
+ #endif
\end{lstlisting}
\vspace{-.5ex}
Value-based features will be referenced in the implementation, acting as a place-holder for a value defined during the configuration process,
as shown in Listing \ref{listing_value}.
\vspace{-.5ex}
\begin{lstlisting}[caption=Referencing to a value feature where the variable DSL will take the value associated with feature DE2104X\_DSL,label=listing_value] 
#define DSL  CONFIG_DE2104X_DSL
\end{lstlisting}
\vspace{-.5ex}

