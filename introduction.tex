%intro

\section{Introduction}
\label{sec:Introduction}
%Highly configurable software and its goals
\textit{Highly configurable software systems} allow end-users to tailor a system to suit their needs and expected operational context.
This is achieved through the development of \textit{configurable} components, allowing systematic reuse and mass-customization \citep{van_gurp_notion_2001}.
The benefits of such development strategies are to reduce the time to market, as mass-customization facilitates the creation of tailored solutions, and improved software quality, as re-used components are tested in various contexts \citep{clements_software_2002}.
Examples of such systems can be found in various domains, such as database management \citep{rosenmuller_fame-dbms:_2008,batory_genesis:_1988}, SOA based systems \citep{kumara_sharing_2013}, 
operating systems \citep{berger_variability_2010}, and a number\footnote{http://splc.net/fame.html}  of industrial
and open source software projects \citep{liebig_analysis_2010} among which the Linux kernel may be the best known.

% architecture of such system - specific characteristics
A constraint of such a development strategy is the  fragmentation of concerns among development artefacts in such a 
way that re-use and customization can be achieved.  
Configuration options, or \textit{features}, play a significant role in a number of inter-related artefacts of different nature.
For systems where variability is mostly resolved at build-time, features will play a role in, at least, the following three spaces \citep{neves_safe_2015,dietrich_understanding_2012}:  
\begin{enumerate}
\item \textit{the variability space} - describing available features and their allowed combinations; 
\item the \textit{implementation space}, comprised of re-usable assets, among which configurable implementation artefacts;
and finally
\item the \textit{mapping space} - relating features to assets and often supported by a build system like Makefiles;  
\end{enumerate}
%Evolution challenges
When such systems evolve, information about feature implementation across those three spaces is actively sought by engineers \citep{heider_case_2012}.
Consistent co-evolution of artefacts is a necessity adding complexity to an already non-trivial evolutionary process \citep{mens_challenges_2005}, 
occurring in both industrial \citep{hellebrand_coevolution_2014} and open-source contexts \citep{passos_coevolution_2015,hunsen_preprocessor-based_2015}.
Inconsistent modifications across the three spaces (variability, mapping, and implementation) may lead 
to the incapacity to derive products, code compilation errors, or dead code \citep{tartler_feature_2011,nadi_mining_2012,abal_42_2014}.

%What we know / can do so far / interest.
Recent studies \citep{passos_coevolution_2015,neves_safe_2015} described typical changes occurring in such systems,
giving insight on how each space could evolve, and revealing the relationship between the various artefacts.
In \citep{passos_dataset_2014}, Passos et al. proposed a dataset capturing the addition and removal of features.

%What is currently not possible in the context of the challenges and w.r.t. what is possible today:
%Manual approaches are not complete : we do not know what are the possible scenarios
%Co-evolution in the large remain unknown: manual approach cannot provide overview or trends.
Unfortunately, the most detailed change descriptions currently available \citep{passos_coevolution_2015,neves_safe_2015}
were obtained using extensive \textit{manual} analysis of commits.
Moreover, those studies focused on specific types of changes, such as addition and removal of features \citep{passos_coevolution_2015}, or product line refinement scenarios \citep{neves_safe_2015}. 
Consequently, the set of co-evolution scenarios documented is limited, and, saved by performing a similar
extensive manual analysis of a large number of commits, the identification of new scenarios remains difficult. 
Finally, the current state of the art offers neither data nor methods to obtain information on 
the prevalence of co-evolution in practice nor the frequency of those specific scenarios over a long period of time.

%How this could be useful
Such feature-related change information is important in various practical scenarios.
\begin{itemize}
\item (\textbf{S1}) A release manager is interested in finding out which commits participated
in the creation of a feature, to build the release notes for instance. 
In such cases, he would be interested in commits introducing the feature, 
and the following ones, adjusting the behaviour of the feature.
\item (\textbf{S2}) A developer introducing a new feature to a subsystem is interested in finding
how similar features were supported by similar subsystems in the past. Then, (s)he needs to look for
changes in those subsystems, involving that such features.
\item (\textbf{S3}) During bug triage, a maintainer is searching for a developer who
might be able to resolve a specific issue. The maintainer would then 
be looking for developers with knowledge in the implementation on the possibly faulty
features.
\item (\textbf{S4}) Researchers focusing on feature-oriented evolution of systems are interested in 
automatically identifying instances of co-evolution patterns or templates, or extending the existing pattern catalog presented by Passos et al.
\citep{passos_coevolution_2015} and Neves et al. \citep{neves_safe_2015}
\item (\textbf{S5}) Researchers working in the field bug prediction for highly configurable systems are
interested in the relationship between variability changes and error-proneness.
A database of detailed feature-related change information could facilitate their work.
\end{itemize}
Unfortunately, given the current state of the art, obtaining the necessary information require extensive manual analysis of changes
and in-depth knowledge of the system under study.

%Overview of the solution we propose and achieved targets
We present in this paper the extension of FEVER (Feature EVolution ExtractoR) \citep{dintzner_fever:_2016}, 
a tool-supported approach designed to automatically extract changes in commits affecting artefacts in all three spaces.
FEVER retrieves the commits from a versioning system and rebuilds a model of each artefact before and after their modification. 
Then it  extracts detailed information on the changes using graph differencing techniques.
Finally, relying on naming conventions and heuristics the changes are aggregated based on the affected feature(s) across all commits in a release.
The resulting data is then stored in a database relating the features and their evolution in each commit.

We then manually compare the data obtained by FEVER and the commits as presented in the source control system
to first evaluate the improvement in terms of change extraction accuracy obtained by the FEVER approach over its previous installment,
and perform a second complete evaluation on a larger set of commits. We use this evaluation to answer the following research questions:
\begin{itemize}
\item RQ1: To what extent is the new version of FEVER more accurate in capturing feature-related changes?
\item RQ2: To what extent does the new version FEVER data match changes performed by developers?
\end{itemize}

%Introduction to the study performed afterwards (extension)
We use the resulting dataset to perform an exploratory study of feature evolution over 15 releases of the Linux kernel.
We focus on the co-evolution of artefacts during feature evolution, in terms of affected spaces, under two different points of view: a feature perspective, focused on feature and the artefacts touched during their evolution, and an author-centric view, focused on commit authors and the spaces affected during maintenance operations.
Using FEVER data, we aim at answering the following two research questions:
\begin{itemize}
\item RQ3: To what extent do artefact in different variability spaces co-evolve during the evolution of features?
\item RQ4: To what extent are developers facing co-evolution over the course of a release?
\end{itemize}

While the tool we built to extract changes is centered on the Linux kernel, the approach itself is applicable to a larger set of systems \citep{berger_survey_2013,hunsen_preprocessor-based_2015}
with an explicit variability model, where the implementation of variability is performed using annotative methods (pre-processor statements in our case), 
and where the mapping between features and implementation assets can be recovered from the build system.

%Key contributions of the work we did
Through this paper, we make the following key contributions:
(1) a model of feature-oriented co-evolving artefacts,
(2) an approach to automatically extract instances of the model from commits,
(3) a dataset of such change descriptions covering 15 recent releases of the Linux kernel history (3.10 to 4.4 in separate databases),
(4) an evaluation of the accuracy of our heuristics showing that we can extract accurately the information out of 87\% of the commits,
(5) we show that most (69.27\%) of features evolve solely through their implementation, and that a majority of authors do not touch other spaces than the implementation space.
Finally, the tool and datasets used for this study are available on our website.\footnote{\url{http://swerl.tudelft.nl/bin/view/NicolasDintzner/}}

This study is an extension of our previous work on co-evolution of artefacts in highly variable systems \citep{dintzner_fever:_2016}.
In this paper, compared to \citep{dintzner_fever:_2016},
we improved the model to better describe complex changes, with additional relationships between artefacts and information on artefact changes. 
We also improved the heuristics use to capture changes, leading to a higher change extraction accuracy.
We also extracted a larger dataset, comprised of more detailed changes and over a longer period of time.
Finally, research questions RQ3 and RQ4, on the quantitative aspect of co-evolution, are entirely new
to this work.

We first provide background information on highly variable systems and the implementation of features in the Linux kernel in \secref{sec:background}.
Then, we present the FEVER approach, its change meta-model and the change extraction process in \secref{sec:approach}.
We evaluate our approach by first comparing the performance of FEVER with its previous version presented in \citep{dintzner_fever:_2016}, and then
provide a complete evaluation, including new change attributes in \secref{sec:evaluation}.
We show the usefulness of FEVER and the collected data in the aforementioned scenarios in \secref{sec:in_practice}.
With the collected data, we perform an exploratory study of co-evolution occurrence in \secref{sec:coevolution}.
We discuss our results and present the threats to the validity of our approach and complete study in \secref{sec:discussion}.
Finally, we present related work in \secref{sec:related} and conclude our work in \secref{sec:conclusion}.
