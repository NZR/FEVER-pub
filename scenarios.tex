%intro

\section{FEVER usage scenarios}
\label{sec:in_practice}
In this section, we illustrate how using FEVER or the data collected using the approach can be of use 
to developers, maintainers and researchers in the scenarios \textbf{S1} to \textbf{S4} mentioned in the \secref{sec:Introduction}.

The FEVER data is stored in a Neo4j graph database.\footnote{http://neo4j.com/} 
Every entity of the FEVER change meta-model is a node of the graph, and the relationships are edges.
Data types are represented using node labels, and attributes are stored as node properties.
The queries presented in this section are written in the Cypher query language.\footnote{http://neo4j.com/docs/stable/cypher-query-lang.html}
It is understood that, in a practical situation, an integration with development tools would be more suitable than 
relying on direct Cypher queries.

\subsection{FEVER for Software Development Activities}

In scenario \textbf{S1}, we consider the work of a release manager
building the release notes. He is interested in highlighting important features, 
and matching those to the commits that participated in their implementation.
The release notes of Linux v3.13 \footnote{http://kernelnewbies.org/Linux\_3.13}
mention the following change ``add[s] option to disable kernel compression'' with a single commit.
Looking at the commit, we know that a new configuration option named ``KERNEL\_\-UNCOMPRESSED'' is introduced.
We can check this with FEVER by querying the commits 
associated with the \textbf{TimeLine} of ``KERNEL\_\-UNCOMPRESSED'' as follows:
\vspace{-.5ex}
\begin{verbatim}
match 
(t:TimeLine)-[]->()<-[]-(c:commit)
where t.name = "KERNEL_UNCOMPRESSED"
return distinct c; 
\end{verbatim}
\vspace{-.5ex}
This query returns two commits. The first commit (id:69f055) mentioned in the release note
is associated with a \textbf{FeatureEdit} entity denoting the addition of a feature.
The second commit (id:2d3c62), occurring a few days later, is also associated with a \textbf{FeatureEdit} entity, but, surprisingly, \emph{removes} the feature.
A check in release v3.14 showed that the feature was never re-introduced. This means that the release notes written by the 3.14 release managers were, in fact, incorrect.
We argue that a dataset such as FEVER would provide release manager with more accurate information on changes that were performed by developers and may have prevented this erroneous entry in the release notes.

In scenario \textbf{S2}, a developer is about to introduce a new driver for a touch-screen 
supporting the power management ``SLEEP'' feature.
The developer might want to know how such support was implemented in other drivers and compare it with its own implementation.
Using FEVER, he queries the database for commits where a new feature (f1) is added (fe.change =``ADDED''),
and interacts with a second feature (f2) whose name is ``PM\_\-SLEEP'' as follows:
\vspace{-.5ex}
\begin{verbatim}
match (f1:TimeLine)-[:FEATURE_CORE_UPDATE]->
             (fe:FeatureEdit)<-[]-(c:commit),
 (c)-[]->()<-[:FEATURE_INFLUENCE_UPDATE]-(f2:TimeLine)    
where f2.name = ``PM_SLEEP''  and fe.change = ``ADDED''
return f1,f2, distinct c;
\end{verbatim}
\vspace{-.5ex}

When ran against database containing commits of release 3.14 of the Linux kernel, this query returns ten results, giving the name of the
newly introduced features, and the commits in which those changes occurred.
Among the results, the developer might notice that feature ``TOUCHSCREEN\_ZFORCE''
and might consider using this as an example to drive his own development.

In this scenario, FEVER is used as a ``recommender'' system to guide the implementation of a new component.
Relying on previous activities to guide further development is a common approach to ease software evolution.
We can name Hipikat \citep{cubranic_hipikat:_2003}, CodeBook \citep{begel_codebook:_2010} as tools aiming for such facilities.
However, such approaches do not take into account the deep structure of the implementation
which FEVER does by breaking artefact changes by feature.
This degree of granularity is particularly interesting for variant-rich system.
For such scenarios, we believe that the information obtained by FEVER would be a valuable addition to existing approaches such as CodeBook,
rather than a replacement.


FEVER can also be of use in our third scenario \textbf{S3} in the context of bug triaging.
Let us consider the bug \#928561 reporting issues with keyboards mentioning that ``\textit{multimedia and macro keys are not working}''.\footnote{https://bugzilla.redhat.com/show\_bug.cgi?id=928561}
The bug report author provide traces and logs pointing to issues with the Linux Human Interface Devices (HID) subsystem.
This issue was fixed by and the patch was introduced in the  kernel in release 3.12.
In the FEVER database for release 3.12, we run the following query to see who among the commit authors committed
the most changes affecting HID related features.
\vspace{-.5ex}
\begin{verbatim}
match (c:Commit)-->()<--(t:TimeLine)
	where t.name=~"(?ism).*HID.*" 
	return distinct (c.author), count(c) 
	order by count(c) desc;
\end{verbatim}
\vspace{-.5ex}
The name of the developer who analyzed and fixed the issue comes first in the results, with 22 commits affecting ``*HID*'' features - 
among which one corresponds to the patch fixing  the keyboard issue.
In second place, we find an official maintainer for three of kernel subsystems with  17 commits, followed by another official maintainer 
for two HID related subsystems and the name of a Linux branch manager with 16 commits each affecting such features.
It is interesting to note that the names of the developers who fixed the issue in question 
are not present in the official maintainers list of the kernel for releases 3.11 nor 3.12. 
Through this scenario, we suggest that the FEVER database can be of use to identify feature expertise,
and possibly facilitate bug triaging \citep{matter_assigning_2009}.
A maintainer in charge of bug triage may use a simple query with information on potentially faulty features
to find which developers can provide insight on an issue or even fix it.

The number of bug reports, and the number of developers makes appropriate bug assignment to developers difficult.
To alievate some of those issues, several approaches have been designed to facilitate the identification of experts capable of fixing a bug \citep{ahsan_automatic_2009,matter_assigning_2009}.
Most of those solutions rely on previous fixes to determine, based on bug report content, who is the most likely to be able to provide an answer to a bug report.
What we propose is to take into account a feature-based expertise, and relate the bug report content with specific features, in order to determine
who is the best suited to fix bugs related to that feature.
This provides an additional type of information, based on fine-grained artefact changes, which can be particularly useful for artefacts at the limit 
of a subsystem where more than one team may be considered as potential fixers.
We do not claim that FEVER could replace existing approaches, but feature-related evolution information could be added to increase the accuracy
of existing techniques.

\subsection{FEVER for Software Engineering Research}

In scenario \textbf{S4}, a researcher in the domain of evolution of highly variable software systems
is interested in the typical structure of feature related changes.
For instance, he would like to observe the occurrences of the introduction of abstract features, in the sense of Thuem et al. \citep{thuem_reasoning_2009}: a feature only exists in the VM.
Using FEVER, we can identify the introduction of such features with this query:
\vspace{-.5ex}
\begin{verbatim}
match 
 (t:TimeLine)-[:FEATURE_CORE_UPDATE]->(f:FeatureEdit)
where 
 not (t)-[:FEATURE_CORE_UPDATE]->(:MappingEdit)
 and not (t)-[:FEATURE_CORE_UPDATE]->(:ArtefactEdit)
 and not (t)-[:FEATURE_INFLUENCE_UPDATE]->(:SourceEdit)
 and f.change="Add"
return t
\end{verbatim}
\vspace{-.5ex}
In release v3.13, this query returns 42 features. 
Because \textbf{TimeLine} entities are regrouping changes across spaces and commits,
we know that those 42 features are indeed abstract, and this is not the result
of a developer who first modified the variability model and in a later commit adjusted the implementation.
The addition of an abstract feature has not yet been described as a co-evolution pattern, and further analysis 
is necessary to fully describe such changes. 
Nonetheless, this illustrates how FEVER can be of use to discover patterns or identify instances of known patterns.
An earlier version of FEVER was used by Sampaio et al. to facilitate the identification of instance of changes affecting certain spaces
in the context of their work on safe evolution templates \citep{sampaio_partially_2016}.

In scenario \textbf{S5}, we consider the work of a researcher focusing on variability related bugs \citep{abal_42_2014} and bug prediction \citep{giger_comparing_2011}.
The data captured by FEVER may reveal information on features involved in bug-fixing commits.
A basic approach would consist in using regular expression on commit messages to identify bug-fixing commits.
Using this, one can identify features involved in bug-fixing commits using the following query:
\vspace{-.5ex}
\begin{verbatim}
match (c:commit)-->()<--(t:TimeLine)
	where not c.message =~ "(?ism).*copyright notices.*" 
			and c.message =~ "(?ism).* bug.*" 
			or c.message =~ "(?ism).* error.*" 
			or  c.message =~ "(?ism).* fix.*" 
			or c.message =~"(?ism).* revert.*" 
return t.name, count(distinct c);
\end{verbatim}
\vspace{-.5ex}
We note that Tian et al. devised a methodology to identify bug-fixing commits in the Linux kernel \citep{tian_identifying_2012}.
Combining such an approach with FEVER should yield more accurate results than the query presented here.
However, with such a simple query, one can identify which features are more error-prone than others.
It would be interesting to see if the number of features involved in a commit influences
the bug-proneness of commits.

Finally, the data provided by German et al. \citep{german_continuously_2015} can be used to track commits over time and across repositories.
Combining this information with the FEVER database would allow us to track feature development across Git repositories, 
and observe how the Linux community collaboratively handles the development of inter-related features.

The implementation of variant-rich system is known to be challenging. Features and their relationships, if misunderstood, can lead to issues such as dead code,
invalid products, or compilation errors.
To mitigate such problems, researchers have to identify the issues, find a way to fix them, and apply them on a number of cases for validation.
We argue here that tools such as FEVER are a convenient way of identifying what changes occurred in commits with respect to features.
Once a researcher has found a scenario where the studied error occurs, one can easily find other scenarios, with similar changes, and observe if the error occurred there as well.
FEVER by itself cannot mitigate such a problem, but constitutes a way to facilitate research in such a domain: by easing the search for a problematic situation, and providing a  quantitative estimate of the occurrences of problematic evolution scenarios.

Given the current accuracy of the FEVER prototype (85\%), a manual review of the changes is necessary to guarantee that the retrieved
changes are all correct. However, it is sufficient to reduce efficiently the number of commits that must be reviewed, and provide a solid starting point
for further manual analysis, as was done by Sampaio et al. \citep{sampaio_partially_2016}.

Conversely, should a developer run a query on FEVER and get no results, there is a small chance that FEVER may have failed to extract such changes (false negative).
In such a situation, the developer might have to rely on Git query instead, but FEVER already provides some information:
the type of change sought by the developer is not common, or the implementation used to support those specific features/constructs are
not what is commonly used in the Linux kernel.










